\documentclass[letterpaper, 11pt]{article}

\usepackage{geometry}
\geometry{hscale=.5, vscale=.8, hmarginratio=2:1, vmarginratio=1:1, marginparwidth=.18\paperwidth, ignoremp}
%\geometry{marginparwidth=.1\paperwidth}

%\usepackage[T1]{fontenc}

\usepackage[explicit]{titlesec}
\titlespacing*{\section}{\dimexpr -\marginparsep-\marginparwidth}{*4}{*1}
\titleformat{\section}[runin]{\large\bfseries\titlerule[.5pt]\filright}{\makebox[1em][c]{\thesection}}{1em}{\parbox[t]{\dimexpr\marginparwidth-2em}{#1}\hskip\marginparsep\mbox{}}[\newline]

%\titlespacing*{\subsection}{\dimexpr -\marginparsep-\marginparwidth}{*4}{*1}
%\titleformat{\subsection}[runin]{\large\bfseries\titlerule[.5pt]\filright}{\makebox[1em][c]{\thesection}}{1em}{\parbox[t]{\dimexpr\marginparwidth-2em}{#1}\hskip\marginparsep\mbox{}}[\newline]

\usepackage{enumitem}
\newlist{steps}{enumerate}{1}
\setlist[steps]{label=Step \arabic*, font=\bfseries, leftmargin=-\marginparsep, itemindent=\marginparsep, align=right}

\usepackage{fancyhdr}
\pagestyle{fancy}
\fancyhf{}
\fancyhfoffset[lh,lf]{\dimexpr\marginparwidth+\marginparsep}
\fancyhf[lh]{UCD EEC 134}
\fancyhf[ch]{}
\fancyhf[rh]{}
%\fancyhf[lf]{left foot}
%\fancyhf[cf]{centre foot}
%\fancyhf[rf]{right foot}
\renewcommand{\footrulewidth}{.4pt}

\usepackage{kantlipsum}

\usepackage{marginnote}

\begin{document}

\title{Lab 1: Elements of Electronic Systems}
\date{}

\maketitle

\section{Objectives}
The main objective of this lab is to understand the basic components of a typical electronic system. Specifically, this include the following:
\begin{enumerate}
\item Understand the functionality and characteristics of linear and switching voltage regulators;
\item Learn how to program a simple micro-controller, the Arduino Uno;
\item Understand the functionality and use of digital-to-analog converters (DAC) and analog-to-digital converters (ADC);
\item Understand the serial programming interface to controlling electronic components;
\item Learn the basic skills of designing and laying out a printed circuit board (PCB);
\item Learn the basic skills PCB assembly involving surface mount (SMD) components.

Be warned that this lab is a fairly aggressive one and it will take a lot of time for you and your group to finish all the reading, the pre-lab assignment, the actual lab, and the reports. It's a good idea to start early!
\end{enumerate}

\section{Prelab}

\subsection{Function Generator}
We start our labs with building a simple function generator that can output triangle, square, sawtooth, and sinusoidal waveforms. This first lab exists for two reasons:
\begin{enumerate}
	\item We need a function generator later when we construct our RF system
	\item This is a simple and somewhat inclusive project to expose you to the basics of using a microcontroller, which later becomes handy when we want to control other aspects of our system. 
\end{enumerate}

There are numerous ways to build a function generator. You can use a 555 timer, a ring of inverters, dedicated oscillator ICs, or the more recent invention of direct digital synthesis (DDS) circuits. A DDS circuit takes advantage of the ever-increasing speed of digital circuits to implement fast waveform generators. It does so by outputing values from a look-up table (LUT), which stores the desired waveform in discretized values, and converting them to an analog signal by a digital-to-analog converter (DAC). 


\reversemarginpar
\marginnote{\textbf{Assignment}}

  \kant[2]

\section{How To Do More Things With Text more thing more thing}
  \kant[2]
  \begin{steps}
    \item First
    \item Second
    \item Third
  \end{steps}

\end{document}